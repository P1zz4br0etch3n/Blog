\documentclass[10pt]{article}

\usepackage{fullpage}

\title{Dokumentation des Go-Projekts}
\author{Von 2416160, 5836402}
\date{}

\begin{document}
\maketitle
	\section{Architekturdokumentation}
	\section{Anwenderdokumentation}
	\section{Inbetriebnahme}
		\subsection{Start ohne Anpassung}
		Der Server kann g\"anzlich ohne Anpassungen gestartet werden.
		Die Standardeinstellungen sehen wie folgt aus:
		
		\subsection{Anpassung der Konfiguration}
		\subsection{Start des Servers}
		\subsection{Serverinformationen via REPL}
		\"Uber einen sog. "Read Eval Print Loop" ist es m\"oglich im laufenden Betrieb Informationen abzurufen.
		Das Kommando "?" zeigt die Befehle an, die akzeptiert werden. Die Befehle sehen folgenderma{\ss}en aus:\\\\
		\begin{tabular}{l|l|l}
			Kommando   & Effekt   & Beschreibung\\
			\hline
			q          & quit     & Beendet den Server\\
			s          & settings & Gibt die aktuell aktiven Einstellungen aus\\
			v          & version  & Gibt die Versionsnummer aus\\
			u          & users    & Gibt alle aktiven User aus (alle Users mit Session)\\
			(r config) & -        & Undocumented: L\"adt die Konfiguration neu
		\end{tabular}
	\section{Beitrag zum Projekt}
		\subsection{2416160}
		\subsection{5836402}
\end{document}